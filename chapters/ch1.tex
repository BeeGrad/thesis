\chapter{INTRODUCTION}\label{introduction}

\pagenumbering{arabic}

Inpainting is the process of reconstructing missing and corrupted regions in an image. Early works in this field started with manually repairing deformed photographs by a talented artists. With the transition from analog photography to digital images, algorithms that can perform this process in a much shorter time started to appear. With the development of deep learning technologies and the increase in image data resources, the use of artificial neural network models has become more popular to solve this problem. These methods perform the image inpainting process completely automatically without the need for people.

Image inpainting can be used for purposes such as removing an unwanted object from the image or restoring a noisy image by filling the missing data in images. Traditional image restoration methods are basically designed to fill missing pixel values in a way that similar to the neighboring pixel values. These methods usually give poor results in cases with large missing areas.

Modern methods are usually trained with millions of images from thousands of different labels using supervised machine learning. For example, generative adversarial network, which has the ability to generate new data, is frequently used in such studies. A neural network alone cannot understand the places that need to be filled in an given input image. Therefore, input image is given along with a mask indicating that the related corrupted parts in the image are missing and need to be filled. Then, output image is generated by passing through various network layers. The missing parts of the original image are filled by taking the masked parts from this generated output image.