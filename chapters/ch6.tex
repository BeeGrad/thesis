\chapter{CONCLUSION}\label{conclution}

Various models trying to achieve the same result by following different methods were examined, applied and compared throughout the the project. The obtained results are presented visually and mathematically in the project. On the other hand, it has been tried to add new methods to image inpainting method.

\section{Possible Applications of this Project}

Image related problems are getting more important each day in our society. Image inpainting is also one of these important image related areas. Image inpainting operation might be needed for various reasons, hence it's possible applications may vary.

\begin{itemize}
  \item Correction of photos: As it is mentioned before, distortions in images may happen in various ways. One of the important usage of the image inpainting is correcting the distorted parts of target image. 
  \item Removing unwanted parts from photos: Image inpainting operation might be needed to remove an unwanted region rather than a distorted area. Many people want to manipulate their photos for a lot of reason such as sharing them online. A lot of softwares are used for this purpose. It is also possible to use deep learning for the same reason.
\end{itemize}

\section{Realistic Constraints}

Training deep learning models are always hard to deal with. Even though it is easy to reach high computational power and huge datasets in recent years, it still take a lot of time to train these deep learning models. One of the main challenges of the image inpainting is to produce visually realistic images. It is possible to get better results mathematically, while output images are not visually realistic.

\subsection{Social, environmental and economic impact}

As it is mentioned before, for purposes like image inpainting, commercial softwares are commonly used and professionals might work in this area. Image inpainting with deep learning could be a powerful alternative for these softwares. On the other hand, getting better results with image inpainting could effect social media usage. 

\subsection{Cost analysis}

Cost of this project is only restricted with training equipments for the networks an engineers' wages.

\subsection{Standards}

IEEE Standart Glossary of Image Processing and Pattern Recognition Terminology is used in this thesis work.

\subsection{Health and Safety Concerns}

There are no health and safety concerns in this project.

\section{Future Work and Recommendations}

Image inpainting is an area that is still actively being developed. As computational power continues to improve and deep learning methods continue to be developed, it is natural to expect better results from image inpainting. In particular to our study, with the development of other areas where deep learning is used, better results can be obtained with multi-disciplinary studies. Our proposed methods could be improved by using bigger datasets or deeper network models which can be trained on better GPUs. Also, it is possible to add a fine tuning layer or a mathmetical operation that smoothes the output of our models.